\documentclass{article}
\usepackage[utf8]{inputenc} % Umlaute
\usepackage[T1]{fontenc} % Pipes
\usepackage[ngerman]{babel} % Deutsche Captions

\title{Phoenix}
\author{Alexander Miller, Daniel Brand \\ \\
		Compiler Construction \\
		Department of Computer Sciences \\
		University of Salzburg}
\date{\today}

\setcounter{tocdepth}{2}

\begin{document}
	\maketitle
	%\newpage
	\tableofcontents
	%\newpage

	\section{Einleitung}
	\subsection{Beschreibung}
	Phoenix ist ein Compiler für eine Teilmenge von C, geschrieben in C.
	Ausgabesprache ist ein Binärformat für einen DLX Interpreter.
	Phoenix unterstützt grundlegende Sprachkonzepte: Type Checking, Lazy Evaluation, Arrays und Records.
	\subsection{Team}
	Phoenix wurde geschrieben von:
	\begin{itemize}
		\item Daniel Brand (1023077)
		\item Alexander Miller (1120667)
	\end{itemize}

	\section{Aufbau}
	Scanner, Parser, Code Generation. Target Machine.

	\section{Scanner}
	Liest ASCII files, arbeitet mit Strings. Von Zeichen zu Token.

	\section{Parser}
	LL(1)-Parser. Von Token zu Syntax-Tree.

	\subsection{Typen}

	\subsection{Arrays}

	\subsection{Records}

	\subsection{Schleifen}

	\subsection{Type Checking}
	Erklärung von Type Checking

	\subsection{Boolean Expressions}

	\subsection{Lazy Evaluation}

	\subsection{Arithmetic Expressions}

	\section{Code Generation}
	Recursive descent.

	\section{Target Machine}
	DLX, 32bit, RISC.

	\subsection{Binärformat}
	Code, Strings, Globals

	\subsection{Ausführung}
	Setzen von GP, Allokieren von Heap, Stack. Semantik der Instruktionen

	\subsection{System IO}
	Printf, File IO.

	\newpage
	\section{Anhang 1: EBNF}
	start ::= top\_declaration || function\_declaration.

	\newpage
	\section{Anhang 2: Unterstütze Instruktionen}
	J
	BSR
	ADDI
	etc.
\end{document}