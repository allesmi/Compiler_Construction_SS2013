\documentclass{article}
\usepackage[utf8]{inputenc}
\usepackage[T1]{fontenc}

\title{Phoenix}
\author{Alexander Miller, Daniel Brand \\ \\
		Compiler Construction \\
		Department of Computer Sciences \\
		University of Salzburg}
\date{\today}

\setcounter{tocdepth}{2}

\begin{document}
	\maketitle
	\newpage
	\tableofcontents
	\newpage

	\section{Einleitung}
	Phoenix ist ein Compiler für eine Teilmenge von C, geschrieben in C. Ausgabesprache ist ein Binärformat für einen DLX Interpreter.

	\section{Aufbau}
	Scanner, Parser, Code Generation. Target Machine.

	\section{Scanner}
	Liest ASCII files, arbeitet mit Strings. Von Zeichen zu Token.

	\section{Parser}
	LL(1)-Parser. Von Token zu Syntax-Tree.

	\section{Code Generation}
	Recursive descent.

	\section{Target Machine}
	DLX, 32bit, RISC.

	\section{Anhang 1: EBNF}
	start ::= top\_declaration || function\_declaration.

	\section{Anhang 2: Unterstütze Instruktionen}
\end{document}