\documentclass[a4paper,12pt]{article}
\usepackage[utf8]{inputenc} % Umlaute
\usepackage[T1]{fontenc} % Pipes
\usepackage[ngerman]{babel} % Deutsche Captions
\usepackage{listings} % source code listings
\usepackage{multicol} % Aufzählungen in mehreren Spalten
\usepackage[margin=1.5in]{geometry}
\lstset{breaklines = true,keepspaces = true,xleftmargin=20pt}

% Optionen für Listings
\lstset{
	language=C,
	inputencoding=utf8,
  literate={ö}{{\"o}}1
           {ä}{{\"a}}1
           {ü}{{\"u}}1
}

\title{Phoenix \\ \large Konstruktion eines Compilers}
\author{Alexander Miller, Daniel Brand \\ \\
		VP Grundlagen Compilerbau \\
		Fachbereich für Computerwissenschaften \\
		Universität Salzburg}
\date{\today}

\setcounter{tocdepth}{2}

\begin{document}
	\maketitle
	%\newpage
	\tableofcontents
	\newpage

	\section{Einleitung}
	Diese Dokumentation beschreibt die Features unseres Compilers Phoenix.
	Wir entschieden uns für C als Sprache, da die geforderten sprachlichen Features damit unmittelbar umgesetzt werden können.

	\subsection{Team}
	Phoenix wurde geschrieben von:
	\begin{itemize}
		\item Daniel Brand (1023077)
		\item Alexander Miller (1120667)
	\end{itemize}

	\section{Features}
	\begin{multicols}{2}
	\begin{itemize}
		\item Basic Types
		\item Arrays
		\item Records
		\item Boolean Expressions
		\item Arithmetic Expressions
		\item Constant Folding
		\item Strings
		\item File I/O
		\item While (auch verschachtelt)
		\item If-Else (auch verschachtelt)
		\item Lazy Evaluation
		\item Prozduren
		\item Call-by-value
		\item Call-by-reference
		\item Scoping von Variablen (Global und lokal)
		\item Type-checking
		\item Self-Scanning
	\end{itemize}
	\end{multicols}

	Nicht implementiert:
	\begin{itemize}
		\item Self-Compilation
		\item Seperate Compilation
	\end{itemize}

	\section{Aufbau}
	Phoenix ist in mehrere Teile aufgeteilt.
	\begin{itemize}
		\item Scanner (in scanner.c)
		\item Parser (in parser.c)
		\item Code Generation (ebenfalls in parser.c)
		\item Target Machine (in margit.c)
	\end{itemize}


	\section{Scanner}
	Der Scanner liest ein Textfile in ASCII-Kodierung ein und fügt Zeichen zu syntaktischen Einheiten (Tokens) zusammen.
	Diese Tokens sind in ihrer Bedeutung eindeutig definiert (Anhang).

	Der Scanner in Phoenix liest solange Zeichen von der Eingabedatei ein, bis er eine Wortgrenze erkennt.
	Danach versucht er, das bis zu diesem Zeitpunkt eingelesene Wort zu kategorisieren und dies als Token an den Parser weiter zu geben.

	Abweichend von C haben wir das static Keyword benutzt, um frühzeitig die Parsingentscheidung zu treffen, ob eine Variable deklariert wird, oder eine Funktion.
	Wenn alle Variablen und Funktionen im selben File sind (bzw. die include-Anweisungen direkt auf die .c-Daten referenzieren), kann der Quellcode ohne Modifikationen auch vom GCC übersetzt werden.

	\section{Parser}
	Der Parser arbeitet nach dem LL(1)-Prinzip. Er liest von Links nach Rechts, evaluiert von Links und hat einen Lookahead von einem Token.

	\subsection{Typen}

	Phoenix unterstützt die Deklaration von Variablen vom Typ Integer und Character.
	Eine Zuweisung einer Variable Character muss ebenfalls mit dem ASCII-Wert des Buchstabens geschehen.
	Eine Zuweisung der Form $ch='a'$ wird nicht unterstützt.
	Implizit können auch Booleans deklariert werden.
	Da aber der boolesche Datentyp in C nicht explizit vorhanden ist, kann keine Variable vom Typ bool angelegt werden.
	Eine Zuweisung eines booleschen Ausdrucks zu einer Integervariable ist jedoch möglich (wird aber vom Parser als Typkonflikt erkannt).

	\subsubsection{Strings}

	Strings können nicht nur zur Laufzeit als Character-Arrays erstellt werden, sondern auch im Quellcode als Konstanten verwendet werden.
	Das ermöglicht Ausgaben mit printf und die im Scanner häufig benutzten stringCompares mit den definierten Keywords.

	\begin{lstlisting}[title=string.c,frame=single]
void printMe(char * string)
{
	printf(string);
}

void main()
{
	printf("Hello");
	printMe("World");
}
	\end{lstlisting}

	\begin{lstlisting}[title=Ausgabe der Target Machine,frame=single]
Phoenix: Margit
===============
Loaded 160 bytes

>'Hello'

>'World'

Execution stopped.
	\end{lstlisting}

	\subsubsection{Arrays}

	Arrays werden mittels malloc zur Laufzeit am Heap erstellt.
	Sie können aus den oben genannten Basic Types bestehen.

	\subsubsection{Records}

	Wie Arrays werden auch Records (structs in C) zur Laufzeit mittels malloc am Heap erzeugt.

	\subsection{Schleifen}

	Phoenix-C erlaubt nur while-Schleifen.
	for-Schleifen können aber damit ebenfalls ausgedrückt werden.

	\subsection{Type Checking}
	Erklärung von Type Checking

	\subsection{Boolean Expressions}

	Boolean Expressions werden üblicherweise als Conditionals in if- oder while-Konstrukten verwendet.

	\subsubsection{Lazy Evaluation}

	Erlaubt das Evaluieren eines booleschen Ausdrucks solange bis das Ergebnis eindeutig bekannt ist.
	Eine Konjunktion wird dadurch vorzeitig verlassen, sobald ein Term false ist.

	\subsection{Arithmetic Expressions}
	Bei den unterstützen arithmetischen Operationen haben wir uns auf die unbedingt benötigten Funktionen beschränkt.

	In arithmetischen Ausdrücken werden konstante Werte zusammengefasst (Constant folding).
	Aus $x + 3 + 5$ wird so zuerst $x+8$ bevor Code generiert wird.

	\section{Code Generation}
	Die Code Generation geschieht während dem Parsing.
	Die Generierung von Code wird so lange wie möglich verzögert, um Optimierungen wie Constant Folding zu erlauben.

	\section{Target Machine}
	Die Target Machine ist eine DLX-Maschine.
	Somit besitzt sie 32 Register mit jeweils 32bit.
	Zusätzlich zu den DLX-Befehlen haben wir neue Instruktionen für Input/Output eingeführt.
	Zusätzlich haben wir das Sichern und Wiederherstellen von Registern bei Prozeduraufrufen als je eine Instruktion ausgedrückt,
	was zu einer Reduktion der Größe der Binärdatei von bis zu 50\% führte.

	\subsection{Binärformat}
	Das erwartete Binärformat ist in folgende Segmente aufgeteilt:
	\begin{itemize}
		\item Code
		\item Strings
		\item Globale Variablen
	\end{itemize}

	Zur Laufzeit werden von der Target Machine Heap und Stack zur Verfügung gestellt.

	\subsection{Ausführung}
	Die Target Machine lädt das Binärfile in den virtuellen RAM.
	Der GP wird an die erste freie Stelle nach dem gelesen File gesetzt.
	Der PC wird auf 1 gesetzt.
	Die Semantik der Befehle ist ansonsten analog zu denen in der Vorlesung.

	\subsection{System IO}
	Wir haben neue Instruktionen für Input und Output eingeführt.
	Diese sind analog zu den benutzten POSIX-Syscalls.
	\begin{itemize}
		\item fopen
		\item fclose
		\item fgetc
		\item fputc
		\item printf
	\end{itemize}
	Die Target Machine verwaltet ein Array an geöffneten Dateien und gibt dem Programm nur den Index zurück.
	Damit umgeht man die Probleme von Zuweisungen von 64bit Filepointern zu 32bit Registern.

	\newpage
	\section{Anhang 1: EBNF}
	
	\begin{lstlisting}	
start = {include_def} {top_declaration}.

include_def = "#include" (string_literal ||
("<" {identifier || "."} ">") ).

top_declaration = type_declaration ";" ||
variable_declaration ";" || function_declaration.

type_declaration = struct_declaration ||
typedef_declaration.

variable_declaration = ["static"] type 
["*"] identifier.

function_declaration = type identifier 
formalParameters (";" || "{" 
[variableDeclarationSequence] {instruction} "}").

struct_declaration = "struct" identifier
"{" {variable_declaration ";"} "}".

typedef_declaration =

type = "int" || "char" || "void" || identifier || 
("struct" identifier).

identifier = letter {letter || digit}.

formalParameters = "(" formalParameter { "," 
formalParameter } ")".

formalParameter = type identifier.

variableDeclarationSequence =
{ variable_declaration ";" }.

instruction = if_else || fclose_func ";" || 
while_loop ||  return_statement ";" || 
printf_func ";" || fputc_func ";" || identifier 
( actualParameters || "=" expression) ";".

if_else = "if" "(" expression ")" "{" 
{ instruction } "}" [ "else" "{" 
{ instruction } "}" ].

fclose_func = "fclose" "(" expression ")".

while_loop = "while" "(" expression ")" "{"
{ instruction } ")".

return_statement = "return" [expression].

printf_func = "printf" "(" expression ")".

fputc_func = "fputc" "(" expression "," 
expression ")".

actualParameters = "(" [expression {
"," expression } ] ")".

expression = simple_expression 
[ ("==" || "<=" || "<" || "!=" || ">" || ">=")
expression ].

simple_expression = ("-" simple_expression) ||
term [ ("+" || "-" || "||") term].

term = factor [ ("*" || "/" || "&") factor].

factor = ("!" factor) || "(" expression ")" || 
integer || string_literal || identifier || 
sizeof_func || malloc_func || fopen_func ||
fgetc_func || fputc_func.

sizeof_func = "sizeof" "(" type ")".

malloc_func = "malloc" "(" expression ")".

fopen_func = "fopen" "(" expression 
["," expression] ")".

fgetc_func = "fgetc" "(" expression ")".

fputc_func = "fputc" "(" expression ","
expression ")".

string_literal = """ string """.

string = letter { letter || digit || " " }.

integer = digit { digit }.

letter = A-Za-z.

digit = 0-9.
	\end{lstlisting}

	\newpage
	\section{Anhang 2: Unterstütze Instruktionen}
	\begin{lstlisting}
NOP 	// Initialwert im Memory
// F1 (1-23)
ADDI	// Addition mit einem konstanten Wert
SUBI	// Subtraktion
MULI	// Multiplikation
DIVI	// Division
MODI	// Modulo
CMPI	// Vergleich mit einem konstanten Wert
LW  	// Laden in ein Register
SW  	// Speichern in Speicherstelle
POP 	// Wert vom Stack holen
PSH 	// Wert auf dem Stack speichern
BEQ 	// Branch wenn Register gleich 0
BGE 	// Branch wenn groesser oder gleich
BGT 	// Branch wenn groesser
BLE 	// Branch wenn kleiner oder gleich
BLT 	// Branch wenn kleiner
BNE 	// Branch wenn ungleich
BR  	// Branch
BSR 	// Branch in Subroutine
MALLOC 	// Allokieren am Heap
RET 	// Zurück zur aufrufenden Prozedur
FOPEN 	// File oeffnen
FGETC 	// Character lesen
FPUTC 	// Character schreiben
// F2 (24-43)
SUB 	// Subtraktion von Registern
MUL 	// Multiplikation
DIV 	// Division
MOD 	// Modulo
CMP 	// Compare
AND 	// Boolesche Konjunktion
OR  	// Boolesche Disjunktion
PRINTF 	// Ausgabe eines Strings
PRINTFI // Ausgabe einer Zahl
ADD 	// Addition von Registern
// F3 (43-63)
JSR 	// Jump in Subroutine
J   	// Jump
TRAP 	// Erfolg
FCLOSE 	// File schliessen
	\end{lstlisting}
\end{document}